\documentclass{article}
\usepackage{graphicx} % Required for inserting images

\title{IF677 - Infraestrutura de Software}
\author{Vinícius Seabra Lago Lima}
\date{April 2023}

\begin{document}

\maketitle

\section{Introdução}

\begin{figure}[ht]
    \centering
    \includegraphics[scale=0.12]{sistopLivro.jpg}
    \caption{Capa do livro Sistemas Operacionais Modernos, 4ª Edição, escrito por A. S. Tanenbaum} \cite{imagemlivro}
    \label{fig: imagemlivro}
\end{figure}
\paragraph{A cadeira Infraestrutura de Software está dividida em dois módulos, primeiramente se estuda sobre Sistemas Operacionais (assuntos como os Processos de um sistema operacional, Memória Virtual e Dispositivos de Entrada/Saída), e depois sobre Sistemas Distribuidos (assuntos como Midlleware e a Concorrência). Os livros na qual a disciplina se baseia são [Sistemas Operacionais Modernos – 4ª Edição, escrito por A. S. Tanenbaum (Figura \ref{fig: imagemlivro})], [Sistemas Operacionais: Projeto e Implementação – 3ª Edição, escrito por A. S. Tanenbaum e A. Woodhull], [Distributed Systems: Concepts and Design -- 3rd/4th Edition, escrito por George Coulouris, Jean Dollimore, Tim Kindberg] e [Real-Time Systems: Design Principles for Distributed Embedded Applications - 2ª  Edição, escrito por H. Kopetz].} 

\section{Relevância}

\begin{figure}[ht]
    \centering
    \includegraphics[scale=0.3]{sistemas-operacionais.png}
    \caption{Grandes sistemas operacionais}
    \label{fig: imagemsistsops}
\end{figure}

\paragraph{A cadeira IF677 (Infraestrutura de Software) tem uma grande impotância no entendimento dos alunos dos sistemas de software básicos de um computador, no caso uma introdução aos sistemas concorrentes e os sistemas operacionais ( Figura \ref{fig: imagemsistsops}). Tendo assim como principal finalidade fazer com que os discentes entendam como funciona os sistemas de software, que proporcionam uma infraestrutura que aplicativos em geral podem interagir com o hardware.}\cite{if677}

\section{Relação com outras disciplinas}

\begin{figure}[ht]
    \centering
    \includegraphics[scale=0.26]{logo-cin.png}
    \caption{Centro de Informática da UFPE}\cite{cin}
    \label{fig: cin}
\end{figure}

\paragraph{A cadeira Infraestrutura de Software tem uma grande relação com outras disciplinas, principalmente com Infraestrutura de Hardware e Infraestrutura de Comunicação, o CIn (Figura \ref{fig: cin}) preza muito para que as disciplinas se complementem. A disciplina de Infraestrutura de Software faz parte junto com Hardware e Comunicação de um trio que é a base para a construção da maioria dos sistemas de computação atuais. Para estar elegível a cursar esta cadeira, tem que ter concluido a cadeira de Introdução a Programação.}

\bibliographystyle{apalike}
\bibliography{ref.bib}

\end{document}
